\documentclass[a4paper]{article}

\usepackage{graphics} % for pdf, bitmapped graphics files
\usepackage{times} % assumes new font selection scheme installed
\usepackage{verbatim}
\usepackage{natbib}
\usepackage{color}
\usepackage{url}
\usepackage{graphicx}
%\usepackage[ruled,linesnumbered,vlined]{algorithm2e}
\usepackage{amsmath,amsfonts,amssymb,bm,amsthm}
\usepackage{listings}

%\lstdefineformat{prompt}{~=\( \sim \)}
%\lstset{basicstyle=\ttfamily,format=prompt}
\lstset{basicstyle=\ttfamily}

%From mitko
\newcommand{\subst}[2]{(#1 \leftarrow #2)}
\newcommand{\proj}[2]{\left.#1\right|_{#2}}

\newcommand{\mbm}[1]{\mbox{\boldmath $#1$}}
%\usepackage{tabularx,colortbl}
\usepackage{bbm} % bbm fonts
\usepackage{subfigure}  % use for side-by-side figures
%\usepackage{hyperref}   % use for hypertext links, including those to external documents and URLs
\usepackage{soul}
\usepackage{cleveref}

%COLORS
\definecolor{gray}{rgb}{0.8,0.8,0.8}\newcommand{\gray}{\color{gray}}
\definecolor{darkgray}{rgb}{0.6,0.6,0.6}\newcommand{\darkgray}{\color{darkgray}}
\definecolor{white}{rgb}{1.0,1.0,1.0}\newcommand{\white}{\color{white}}
\definecolor{blue}{rgb}{0.0,0.0,1.0}\newcommand{\blue}{\color{blue}}
\definecolor{mygray}{rgb}{0.8,0.8,0.8}\newcommand{\mygray}{\color{mygray}}
\definecolor{mydarkgray}{rgb}{0.6,0.6,0.6}\newcommand{\mydarkgray}{\color{mydarkgray}}
\definecolor{mydarkgrayA}{rgb}{0.3,0.3,0.3}\newcommand{\mydarkgrayA}{\color{mydarkgrayA}}
\definecolor{mywhite}{rgb}{1.0,1.0,1.0}\newcommand{\mywhite}{\color{mywhite}}

% Federico's gray box
\newcommand{\graybox}[1]{\vspace{0.3cm}\noindent
  \fcolorbox{black}{mygray}{\parbox{0.985\textwidth}{#1}}
\vspace{0.3cm}}

% Federico's gray box w/ title
\newcommand{\grayboxt}[2]{\vspace{0.3cm}\noindent
%  \fcolorbox{black}{mygray}{\parbox{0.985\textwidth}{{\bf #1} #2}}
  \fcolorbox{black}{mygray}{\begin{minipage}{0.985\textwidth}{\bf #1} #2\end{minipage}}
\vspace{0.3cm}}

% Mitko's gray box
\newcommand{\gbox}[1]{
  \begin{center}
    \fcolorbox{black}{gray}{
      \begin{minipage}[b]{0.98\textwidth}
        \begin{center}
          %\vspace{2mm}
          \begin{minipage}{0.97\textwidth}
            #1 
          \end{minipage}
          \vspace{2mm}
        \end{center}
      \end{minipage}
    }
  \end{center}
}

% Mitko's white box
\newcommand{\wbox}[1]{
  \begin{center}
    \fcolorbox{black}{white}{
      \begin{minipage}[b]{0.98\textwidth}
        \begin{center}
          %\vspace{2mm}
          \begin{minipage}{0.97\textwidth}
            #1 
          \end{minipage}
          \vspace{2mm}
        \end{center}
      \end{minipage}
    }
  \end{center}
}


\newtheorem{theo}{Theorem}
\newtheorem{defin}{Definition}
\newtheorem{reduction}{Reduction}
\newtheorem{prop}{Property}
\newtheorem{remark}{Remark}
\newtheorem{lemma}{Lemma}
\newtheorem{coro}{Corollary}
\newtheorem{scen}{Scenario}
\newtheorem{ex}{Exercise}

\DeclareMathOperator*{\argmin}{arg\,min}
\DeclareMathOperator*{\argmax}{arg\,max}

\title{Advanced Artificial Intelligence (DT4048)\\{\em Lab 2: Temporal Reasoning}}

\author{Masoumeh Mansouri, Federico Pecora\\Center for Applied Autonomous Sensor Systems\\\"Orebro University, SE-70182 Sweden\\\url{{masoumeh.mansouri, federico.pecora}@oru.se}}

\date{Fall term 2014 (HT2014)}

\begin{document}

\maketitle

\section{Setup}

svn update

\section{Qualitative Temporal Reasoning}

\section{TCSP}
you can create a variable for a TCSP as follows:

{\tt timePointA} 

\begin{lstlisting}

MultiTimePoint timePointA 
	= (MultiTimePoint)groundSolver.createVariable();

\end{lstlisting}

\noindent The code below shows that the distance between the origin time point and {\tt timePointA} are either (5-7) or (4-6).

\begin{lstlisting}

DistanceConstraint distanceBetweenOriginandA = 
	new DistanceConstraint(
		new Bounds(5, 7), new Bounds(4, 6));

distanceBetweenOriginandA.setFrom(source);
distanceBetweenOriginandA.setTo(timePointA);

\end{lstlisting}

\noindent The distance constraint is added to the constraint solver as follows:

\begin{lstlisting}

groundSolver.addConstraints(
	new DistanceConstraint[] {distanceBetweenOriginandA});
\end{lstlisting}



In a restaurant, we have both a human waiter and a robot waiter. A guest enters the restaurant and orders a coffee. The human waiter takes (5-7) min to make a coffee and the robot waiter prepares the coffee between (8-10) min. In this restaurant, this is the robot waiter serving a coffee. The trip form the counter where the coffee prepared to any guest table is either (4-10) if it navigates through the tables and guests or (6-8) minutes if it choose a fixed predefined path. The serving coffee task is fully accomplished when the robot bring a sugar pot to guest a table after bringing a coffee. In order to avoid coffee getting cold, the sugar pot should be served at most 12 minutes after coffee prepared. The whole serving time (i.e., waiting time for both coffee and sugar) should not exceed than 15 minutes. 

The scenario above can be a result of an autonomous decision process determining what to do for serving a coffee. We are interested to verify this sequence of actions mentioned above is feasible with respect to all the temporal constraints. 



{\ex{}\label{ex:ex1}  
Model the scenario above as a TCSP. 
\vspace{0.1cm}}

{\ex{}\label{ex:ex1}  
What are the possible solutions? 
\vspace{0.1cm}}


{\ex{}\label{ex:ex1}  
Look at the branches shown in {\tt SearchTreeFrame} window, and extract one possible scenario.
\vspace{0.1cm}}


{\ex{}\label{ex:ex1}  
The restaurant increases the waiting time by at most 20 min. Model this fact a constraint and see the new set of solutions. There are solutions other than earliest and latest time solution. Extract a solution that is different form the earliest and latest solution (Implement constraint that reflect these choices).


\vspace{0.1cm}}


\section{STP}

%\bibliographystyle{apalike}
%\bibliography{metaCSP}

\end{document}

